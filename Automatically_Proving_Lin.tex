\documentclass{article}
\usepackage{ulem}
\usepackage{xcolor}
\newcommand{\Cr}{\color{red}}
\newcommand{\B}{\bfseries}
\newcommand{\M}{\mdseries}


\begin{document}
\noindent
Automatically Proving Linearizability\\
$<$Viktor Vafeiadis$>$ \\\\
{\color{blue}{\large{}Q:}\\
1.What is constructor ? \\
2.$s(h)$ is a history corresponds state s ? \\
3.Logical state VS concrete state ?
}
\\\\
\bfseries{Abstract} :\mdseries{}
Presents a practical automatic vefication procedure for proving Linearizability (aatomicity and functional correctness).\\
Linearization points could be hard to point out : \uline{the can depend on a non-local boolean condition and can even reside within other concurrently executing threads }. Brute force search is infeasible .\\


However ,in practice , observation point out that these complicated linearization points arise only in({\color{blue}{is it true only in ?}}) operation executions that {\color{red} {do not} logically update the liberary's shared state}, specification of which was called 
{\color{red}pure specification}. \\


To verify such operations, certain 'pure linearization checker' derived from the specification used to instruments the library code , and runs a suitable powerful abstract interpreter to validate that there are {\color{red}{no assertion violations}} .\\
\colorbox{green!35!blue!24}{ 
\begin{tabular}{|l|}\hline
Work related: \\
1.Line-up \\
2.MC L via refinement.\\
3.Deriving L fine-grained concurrent objects .\\
\hline
\end{tabular}
}\newline
1.On one thing, above do not prove correctness; merely check short execution traces of small number of threads .\\
2.On the positive side , they do not \uline{require linearization points to be annoted} . \newline


Co-methods: static analyses(shape analyses), they require the linearization points SPECIFED, although they work for an unbounded number of threads . \newline
\colorbox{green!35!blue!24}{
\begin{tabular}{|l|}\hline
Work related: \\
1.Comparision under abstraction for verifying linearisability. \\
2.Thread quantification for concurrent shape analyses .\\
3.Shape-value abstraction for verifying linearisability .\\
\hline
\end{tabular}
}\newline
\\\\

{\large{}\bfseries{Definitions}}\newline
1. Lirary ,{\it{A}} consists of a {\it{constructor, $A_{init}$}}, and a number of {\it{methods},$A_{1}$ ... $A_{n}$} \newline
2. Client : Program that calls the lirary's constructor once in its initial sequential phase ,and then call any number of the methods ;\\
{(\color{red}Annotation: we say that a history is sequential if none of its methods overlap in time)} \\

\noindent
{\B{}New linearization definition :}
A library {\it{A}} is linearizable with respect to a specification {\it{B}} iff for all instrumented clients {\it{C}} and every state s , {\it{if($s_{init},s)\in C[A]^*$}}, then there exists a state $s'$ such that $(s_{init}', s')\in C[B]^*$ and $s(h) = s'(h)$, where $s_{init} \ and\  s_{init}'$ are the initial states after calling $A_{init}\ and\  B_{init}$ respectively .
\\
\colorbox{red!35!green!24}{
\begin{tabular}{l}\hline
$*********************$ \\
Use syntatic equality on the recorded histories \\

$<What's\  the\  meaning ?>$ \\
$*********************$\\
\hline
\end{tabular}
}\\\\
3 : \\
Pure executions : executions that do not modify abstract state ;\\
Effective executions : executions that modify abstract state ;
\\\\
\begin{tabular}{l}\hline
Basically, when a verifier checks one thread with a compositional verification technique,
\\ it has a model of what updates all the other threads can do and the effect on the current thread .\\\hline
\end{tabular}
\\\\

The verification procedure is divided in two phase :\\
\begin{tabular}{l}\hline
Phase 1 instrumentes the constructor of the library, computes the pure checkers for each operations and\\ generates
a set of candidate linearization points assignments .\\
Phase 2 iterates over the set to check is any of these assignments valid(if valid then $'Success'$)\\
\hline
\end{tabular}\\\newline

{\B{Preparation Phase}} \\1,generate the library's constructor: $iop_{init} \leftarrow {(op_{init}; spec_{init})}$ \\
2, generate pure checkers : $check_i \leftarrow {GENERATEPURECHECKER(spec_i)}$

{\it{}\Cr for {\B{enqueue}} it's simply the empty command, and for {\B{tryDequeue}}\\

if(*){assume(AQ == @empty); can$\_$return[EMPTY]=true;}}
\newline
3, generate linearization points .\\\newline

{\B{Checking Phase}}.



\end{document}





















